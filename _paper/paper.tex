% Options for packages loaded elsewhere
\PassOptionsToPackage{unicode}{hyperref}
\PassOptionsToPackage{hyphens}{url}
%
\documentclass[
]{article}
\usepackage{lmodern}
\usepackage{amssymb,amsmath}
\usepackage{ifxetex,ifluatex}
\ifnum 0\ifxetex 1\fi\ifluatex 1\fi=0 % if pdftex
  \usepackage[T1]{fontenc}
  \usepackage[utf8]{inputenc}
  \usepackage{textcomp} % provide euro and other symbols
\else % if luatex or xetex
  \usepackage{unicode-math}
  \defaultfontfeatures{Scale=MatchLowercase}
  \defaultfontfeatures[\rmfamily]{Ligatures=TeX,Scale=1}
\fi
% Use upquote if available, for straight quotes in verbatim environments
\IfFileExists{upquote.sty}{\usepackage{upquote}}{}
\IfFileExists{microtype.sty}{% use microtype if available
  \usepackage[]{microtype}
  \UseMicrotypeSet[protrusion]{basicmath} % disable protrusion for tt fonts
}{}
\makeatletter
\@ifundefined{KOMAClassName}{% if non-KOMA class
  \IfFileExists{parskip.sty}{%
    \usepackage{parskip}
  }{% else
    \setlength{\parindent}{0pt}
    \setlength{\parskip}{6pt plus 2pt minus 1pt}}
}{% if KOMA class
  \KOMAoptions{parskip=half}}
\makeatother
\usepackage{xcolor}
\IfFileExists{xurl.sty}{\usepackage{xurl}}{} % add URL line breaks if available
\IfFileExists{bookmark.sty}{\usepackage{bookmark}}{\usepackage{hyperref}}
\hypersetup{
  pdftitle={Estimating the time-varying reproduction number of SARS-CoV-2 using national and subnational case counts},
  hidelinks,
  pdfcreator={LaTeX via pandoc}}
\urlstyle{same} % disable monospaced font for URLs
\usepackage[margin=1in]{geometry}
\usepackage{graphicx,grffile}
\makeatletter
\def\maxwidth{\ifdim\Gin@nat@width>\linewidth\linewidth\else\Gin@nat@width\fi}
\def\maxheight{\ifdim\Gin@nat@height>\textheight\textheight\else\Gin@nat@height\fi}
\makeatother
% Scale images if necessary, so that they will not overflow the page
% margins by default, and it is still possible to overwrite the defaults
% using explicit options in \includegraphics[width, height, ...]{}
\setkeys{Gin}{width=\maxwidth,height=\maxheight,keepaspectratio}
% Set default figure placement to htbp
\makeatletter
\def\fps@figure{htbp}
\makeatother
\setlength{\emergencystretch}{3em} % prevent overfull lines
\providecommand{\tightlist}{%
  \setlength{\itemsep}{0pt}\setlength{\parskip}{0pt}}
\setcounter{secnumdepth}{-\maxdimen} % remove section numbering

\title{Estimating the time-varying reproduction number of SARS-CoV-2 using
national and subnational case counts}
\author{}
\date{\vspace{-2.5em}}

\begin{document}
\maketitle

\textbf{Authors:} Sam Abbott *, Joel Hellewell *, Robin N Thompson,
Katharine Sherratt, Hamish P Gibbs, Nikos I Bosse, James D Munday,
Sophie Meakin, Emma L Doughty, June Young Chun, Yung-Wai Desmond Chan,
Flavio Finger, Paul Campbell, Akira Endo, Carl A B Pearson, Amy Gimma,
Tim Russell, CMMID COVID modelling group, Stefan Flasche, Adam J
Kucharski, Rosalind M Eggo, Sebastian Funk

* contributed equally

\hypertarget{abstract}{%
\section{Abstract}\label{abstract}}

\textbf{Background:} Assessing temporal variations in transmission in
different countries is essential for monitoring the epidemic, evaluating
the effectiveness of public health interventions and estimating the
impact of changes in policy.

\textbf{Methods:} We use case and death notification data to generate
daily estimates of the time-varying reproduction number globally,
regionally, nationally, and subnationally over a 12-week rolling window.
Our modelling framework, based on open source tooling, accounts for
uncertainty in reporting delays, so that the reproduction number is
estimated based on underlying latent infections.

\textbf{Results:} Estimates of the reproduction number, trajectories of
infections, and forecasts are displayed on a dedicated website as both
maps and time series, and made available to download in tabular form.

\textbf{Conclusions:} This decision-support tool can be used to assess
changes in virus transmission both globally, regionally, nationally, and
subnationally. This allows public health officials and policymakers to
track the progress of the outbreak in near real-time using an
epidemiologically valid measure. As well as providing regular updates on
our website, we also provide an open source tool-set so that our
approach can be used directly by researchers and policymakers on
confidential data-sets. We hope that our tool will be used to support
decisions in countries worldwide throughout the ongoing COVID-19
pandemic.

\textbf{Keywords}: Covid-19, SARS-CoV-2, surveillance, forecasting,
time-varying reproduction number

\hypertarget{introduction}{%
\section{Introduction}\label{introduction}}

The coronavirus disease 2019 (COVID-19) pandemic that emerged in
December 2019 has since spread to over 100 countries in every continent
except Antarctica. While some information on the progress of an outbreak
in a given country can be gained from the reported numbers of confirmed
cases and deaths, these numbers can obscure changes in the underlying
dynamics of the outbreak due to delays between infection and the
eventual reporting of a case or death. Accounting for the uncertain
delays from infection to symptom onset, and the uncertain delays from
symptom onset to hospital admission, diagnostic testing or potential
death, followed by further delays until data are recorded in official
statistics, requires the use of specific statistical methods for
handling right-truncated data {[}1--3{]},uncertainty, and the creation
of a ``nowcast'' {[}4,5{]} (an estimate of the current number of newly
infected or symptomatic cases).

A method for tracking the progress of an outbreak is to measure changes
in the time-varying reproduction number (effective reproduction number),
which represents the average number of secondary infections generated by
each new infectious case {[}6--8{]}. This approach can be advantageous
compared to monitoring numbers of newly reported or symptomatic cases
since, in principle, reproduction number estimates reflect variations in
transmission intensity. Due to the delays in disease progression,
recorded numbers of newly notified or symptomatic cases will increase or
decrease for a period after transmissibility has reduced or increased,
respectively. Monitoring changes in the time-varying reproduction can
account for this delay and reveals variations in transmissibility that
are not clear when using only reported cases.

This paper outlines the methods used to produce the website
(\url{https://epiforecasts.io/covid/}), and data resource {[}9{]}, we
have developed that presents real-time estimates and forecasts of
reported cases by date of infection and the respective time-varying
reproduction numbers globally, regionally, nationally and subnationally
for Covid-19. This website relies on methods implemented in the
\texttt{EpiNow2} R package and data aggregated in the
\texttt{covidregionaldata} R package, both developed by the authors
{[}10,11{]}. Our estimates overcome some of the limitations of naive
implementations that derive estimates for the reproduction number
directly from numbers of reported cases without adjusting (or with only
partial adjustments) for the delay from infection to symptom onset or
from onset to notification. Our approach also incorporates multiple
sources of uncertainty that if excluded can bias estimates. The code
that creates and updates the website is open source, and documented for
use by others, allowing policymakers and researchers to run analyses
using confidential data. The methods outlined in this paper and
corresponding code base are under development, and new versions of this
live article will be released alongside changes to the methods to create
a record of the methodology used throughout the pandemic.

\hypertarget{methods}{%
\section{Methods}\label{methods}}

\hypertarget{data}{%
\subsection{Data}\label{data}}

We use daily counts of confirmed cases and deaths reported by the
European Centre for Disease Control from the last 12 weeks for all
analyses conducted at the national level {[}11,12{]}. To estimate the
delay from symptom onset to reporting (once confirmed with a positive
laboratory test), we use all cases from a publicly available linelist
for which onset and notification dates are available {[}11,13{]}. This
linelist combines all known linelist data from over 100 countries at the
time of writing. Countries are only included in the reported estimates
if within the last 12 weeks they have fewer than 14 days with non-zero
case counts. This restriction reduces the likelihood of spurious
estimates for countries with limited transmission or case ascertainment.

For sub-national analyses, the data is aggregated using the
\texttt{covidregionaldata} R package developed by the authors.
Individual data sources are reported on the respective pages of our
website. The data are fetched from government departments or from
individuals who maintain a data source if no official data are
available. Similarly to national estimates, subnational areas are only
included if they report at least 14 days with non-zero cases in the last
12 weeks.

All analyses described below are run daily for each national or
subnational entity under consideration. An automated timestamp is used
to evaluate if data has been updated since the last time estimates were
made in order to avoid repeatedly estimating based on the same data.

\hypertarget{delays-between-case-onset-and-report}{%
\subsection{Delays between case onset and
report}\label{delays-between-case-onset-and-report}}

To estimate the reporting delay (i.e the delay between onset and case
report or death) with appropriate uncertainty, we fit a log-normal
distribution, using use the statistical modelling program stan
{[}10,14{]}, to 100 subsampled bootstraps (each with 250 samples drawn
with replacement) of the available delay data. Accounting for left and
right censoring occurring in the data as each date is rounded to the
nearest day and truncated to the maximum observed delay. There was
insufficient data available on the various reporting delays to estimate
spatially- or temporally-varying delays whilst also accounting for the
biases induced by the growth rate of reported cases, so they were
considered to be static over the 12 weeks of data considered each day.

This results in an onset to case report delay distribution with a mean
of 6.5 days and a standard deviation of 17 days and an onset to death
report delay distribution with a mean of 13.1 days and a standard
deviation of 11.7 days. For computational reasons the maximum allowed
delay is set to be 30 days. Dataset specific estimates are detailed on
the respective country pages. Estimated delays are routinely updated as
new data becomes available.

As data may also be right truncated due to unrecorded delays (i.e the
delay between a case report and its appearance in an aggregated data
set) we truncate all time-series to exclude the last 3 days of data,
based on qualitative inspection of the stability of case counts in the
datasets used.

\hypertarget{estimating-the-time-varying-reproduction-number-and-nowcasting-reported-infections}{%
\subsection{Estimating the time-varying reproduction number and
nowcasting reported
infections}\label{estimating-the-time-varying-reproduction-number-and-nowcasting-reported-infections}}

We estimated the instantaneous reproduction number (\(R_t\)) using the
\texttt{EpiNow2} R package (version 1.2.1) {[}10{]} on the last 12 weeks
of available data, discarding estimates from the first 14 days globally,
for United Nation regions, nationally, and subnationally for 10
countries. The instantaneous reproduction number represents the number
of secondary cases arising from an individual showing symptoms at a
particular time, assuming that conditions remain identical after that
time, and is therefore a measure of the instantaneous transmissibility
(in contrast to the case reproduction number - see Fraser (2007) {[}8{]}
for a full discussion). \texttt{EpiNow2} implements a Bayesian latent
variable approach using the probabilistic programming language Stan
{[}14{]}, which works as follows. The initial number of infections were
estimated as a free parameter with a prior based on the initial number
of cases, or deaths, respectively. The initial, unobserved, growth rate
was estimated from the first 7 days of reported data. This was used as a
prior (normal with standard deviation 0.2) to estimate latent infections
prior to the first reported case using a log linear model. For each
subsequent time step, previous imputed infections (\(I_{t-1}\)) were
summed, weighted by an uncertain generation time probability mass
function (\(w\)), and combined with an estimate of \(R_t\) to give the
incidence at time \(t\) (\(I_t\)) {[}6,7,10{]}. We used a log normal
prior for the reproduction number (\(R_0\)) with mean 1 and standard
deviation 0.2 reflecting our current belief that \(R_t\) is likely to be
centred around 1 in most of the world, with public health interventions
and individual behaviour combining to prevent it from growing
significantly larger for sustained periods. This contrasts with our
earlier approach which was to use a gamma prior with a of mean 2.6 and
standard deviation 2. This was based on early estimates for the basic
reproduction number from the initial stages of the outbreak in Wuhan
{[}15,16{]} with long tails to allow for differences in the reproduction
number between countries. The infection trajectories were then mapped to
mean reported case counts (\(D_t\)) by convolving over an uncertain
incubation period and report delay distribution (convolved into
\(\xi\)). Observed reported case counts (\(C_t\)) were then assumed to
be generated from a negative binomial observation model with
overdispersion \(\phi\) (using 1 over the square root of a half normal
prior with mean 1) and mean \(D_t\), multiplied by a day of the week
effect with an independent parameter for each day of the week
(\(\omega_{(t \mod 7)}\)). Temporal variation was controlled using an
approximate Gaussian process {[}17{]} with a squared exponential kernel
(\(GP\)). In mathematical notation,

\begin{align}
  R_{t} &\sim R_{t-1} \times \mathrm{GP} \\
  I_t &= R_t \sum_\tau w_\tau I_{t - \tau} \\
  D_t &= \sum_\tau \xi_\tau I_{t-\tau} \\ 
  C_t &\sim \mathrm{NB}(D_t \omega_{(t \mod 7)} , \phi)
\end{align}

The parameters of the Gaussian process kernnl were estimated during
model fitting with the following priors. The length scale was given a
lognormal prior with a mean of 21 days and standard deviation of 7 days
truncated to be greater than 3 days and less than the length of the
data. The prior on the magnitude was standard normal. Each time series
was fit independently using Markov-chain Monte Carlo (MCMC). A minimum
of 4 chains were used with a warmup of 400 each and 4000 samples post
warmup. Convergence was assessed using the R hat diagnostic {[}14{]}.

We used an estimate of the generation time sourced from {[}18{]} but
refit using a log-normal incubation period with a mean of 5.2 days (SD
1.1) and SD of 1.52 days (SD 1.1) {[}19{]} rather than the incubation
period used in the original study (code available here:
\url{https://github.com/seabbs/COVID19}). This resulted in a distributed
generation time with mean 3.6 days (standard deviation (SD) 0.7), and SD
of 3.1 days (SD 0.8) for all estimates. The incubation period estimate
was also used to convolve from unobserved infections to unobserved
onsets in the model. See {[}10{]} for further details on the approach.

\hypertarget{estimating-the-daily-growth-rate-and-doubling-time}{%
\subsection{Estimating the daily growth rate and doubling
time}\label{estimating-the-daily-growth-rate-and-doubling-time}}

We estimated the rate of spread (\(r\)) by converting our \(R_t\)
estimates using an approximation derived in {[}20{]}. The doubling time
was then estimated by calculating \(\text{ln}(2) \frac{1}{r}\) for each
estimate of the rate of spread.

\hypertarget{estimated-change-in-daily-cases}{%
\subsection{Estimated change in daily
cases}\label{estimated-change-in-daily-cases}}

We defined the estimated change in daily cases to correspond to the
proportion of reproduction number estimates for the current day that are
below 1 (the value at which an outbreak is in decline). It was assumed
that if less than 5\% of samples were subcritical then an increase in
cases was definite, if less than 40\% of samples were subcritical then
an increase in cases was likely, if more than 60\% of samples were
subcritical then a decrease in cases was likely and if more than 95\% of
samples were subcritical then a decrease in cases was definite. For
countries/regions with between 40\% and 60\% of samples being
subcritical we assumed that cases were approximately stable.

\hypertarget{the-effect-of-changes-in-testing-procedure}{%
\subsection{The effect of changes in testing
procedure}\label{the-effect-of-changes-in-testing-procedure}}

The results presented here are sensitive to changes in COVID-19 testing
practices and the level of effort put into detecting COVID-19 cases,
e.g.~through contact tracing. For example, if numbers of incident
infections remain constant but a country begins to find and report a
higher proportion of cases, then an increasing value of the reproduction
number will be inferred. This is because all changes in the number of
cases are attributed to changes in the number of infections resulting
from previously reported cases and are not assumed to be a result of
improved testing and surveillance. On the other hand, if a country
reports a lower proportion of cases because a lower number of tests are
performed (which can happen if reagents required for testing are no
longer available, for example) or the surveillance system captures a
lower proportion of infections, then the model will attribute this to a
drop in the reproduction number that may not be a true reduction. In
order for our estimates to be unbiased not all cases have to be
reported, but the level of testing effort (and therefore the proportion
of detected cases) must be constant {[}21{]}. This means that, whilst a
change in testing effort will initially introduce bias, this will be
reduced over time as long as the testing effort remains consistent from
this point onwards.

Countries may also change the focus of their surveillance over the
course of the outbreak. They may initially focus on identifying
travellers returning from areas of known COVID-19 transmission and
performing contact tracing on the contacts of known cases. As the
outbreak evolves this may change to passive surveillance at hospitals.
Here, the case definition may also change from tests based on polymerase
chain reaction (PCR) to diagnoses based on symptoms and computed
tomography (CT) scans. In the future, different kinds of COVID-19 tests
may be deployed that could influence results, such as tests that detect
both active and past infections.

\hypertarget{forecasting-the-reproduction-number-and-case-counts-by-date-of-infection}{%
\subsection{Forecasting the reproduction number and case counts by date
of
infection}\label{forecasting-the-reproduction-number-and-case-counts-by-date-of-infection}}

We forecast the time-varying effective reproduction number over a 14-day
time horizon by assuming it remains the same as the last estimated
\(R_t\). The reproduction number forecast is then transformed into a
case forecast using the \texttt{EpiNow2} model outlined in the previous
section {[}10{]}. These forecasts are indicative only and should not be
considered with a weight equal to the real-time estimates. Changes in
contact rates, mobility, and public health interventions are not
accounted for which may lead to significant inaccuracy.

\hypertarget{reporting}{%
\subsection{Reporting}\label{reporting}}

We report the median and 90\% credible intervals for all measures with
20\%, 50\% and 90\% credible intervals shown in figures. The analysis
was conducted independently for all regions and is updated daily as new
data becomes available. To highlight the proportion of cases that have
yet to be reported (due to correcting for right truncation), we show a
cut-off in figures based on the mean of all delays. Values prior to this
point are defined as estimates with values past this point being defined
as estimates based on partial data. In reality, this is a continuum with
estimates closer to now progressively being based on less data and
therefore becoming increasing uncertain. All estimates are available as
downloadable files in csv format under an open-source license for use
elsewhere {[}9{]}. The scheduling framework used to update our estimates
is also available under an open-source license
(\url{https://github.com/epiforecasts/covid-rt-estimate}).

\hypertarget{website-summarised-estimates-and-interactivity}{%
\subsection{Website, summarised estimates, and
interactivity}\label{website-summarised-estimates-and-interactivity}}

We use Rmarkdown templates and the distill framework to generate
webpages summarising these estimates {[}22,23{]}. The \texttt{RtD3}
package is used to provide interactive visualisations of all estimates
{[}24{]}. Estimates by country are provided on a dedicated static page
along with global, and regional, summaries. More detailed subnational
estimates are available for over 10 countries in an flexible framework
into which additional subnational estimates will be added as more data
becomes available.

\hypertarget{results}{%
\section{Results}\label{results}}

Estimates of the reproduction number, trajectories of infections, and
forecasts are displayed on a dedicated website as both maps and time
series, and made available to download in tabular form.

\hypertarget{discussion}{%
\section{Discussion}\label{discussion}}

We provide a centralised resource which generates comparable daily
estimates of the time-varying reproduction number and a daily nowcast of
the number of cases newly infected derived using a standardised method.
We account for the delay between infection and case notification and
include all sources of quantifiable uncertainty. This resource may be
useful for policymakers to track the progression of the COVID-19
outbreak and evaluate the effectiveness of intervention measures. As new
data become available, we will include sub-national estimates for
additional countries, and provide additional support for public health
agencies or researchers interested in applying our methods to their
data.

There are several advantages associated with our approach. Firstly,
reported counts are the only data required, which allows our approach to
be used in a wide variety of contexts. It can be applied separately to
counts of cases, hospital admissions, deaths or other metrics as long as
appropriate delay distributions are used {[}25{]}. As our methodology is
applied across a range of geographies our estimates can be compared
without having to consider differences in the underlying approach (even
if differences in testing should still be accounted for as discussed
below). Finally, we have constructed our approach using open source
tools and all of our code, raw data, and results are available online
and developed with other users in mind. This means our methods can be
readily applied by others to non-public data and be fully evaluated by
end users.

Our approach is also subject to several limitations. Firstly, the model
requires that the proportion of infections that are notified is constant
over the 12 weeks considered. In other words, it requires consistency in
the focus of the surveillance method, level of effort spent on testing,
and case definition. Yet it is often the case that the level of
under-reporting in a country changes over the course of an outbreak
{[}21{]}. However, it should be noted that any changes in surveillance
testing procedures will only bias the estimates temporarily if they
begin to remain consistent again after they have changed. How long the
bias remains in the reproduction number estimates will depend on the
serial generation time and delay distributions, as well as the length
scale of the Gaussian process used in the reproduction number estimation
process. The impact of testing and other reporting biases vary between
measures of transmission (test positive cases, hospital admissions, test
positive deaths) {[}25{]}. For this reason we include estimates based on
reported deaths and provide tooling to allow estimates to be produced
for alternative datasets. In theory, estimates from disparate sources
should be comparable using our approach, however if they in fact
represent different sub-populations then there may be variation between
them that can potentially be usefully interpreted.

In addition, the model is limited by how representative the delay that
we use from infection to notification distribution is for a given
location. As there is limited data to assess this, we estimate a
bootstrapped global delay distribution using the combined data from
every country. In particular, the delay from onset to notification can
especially impact the upscaling of cases by date of onset that accounts
for cases that have onset but not yet been reported. If the true delay
from onset to notification for a given country is shorter than our
global delay, then we will overestimate onset case numbers, and vice
versa for true delays longer than the distribution we used.
Additionally, estimates of the reporting delay distribution are known to
be biased early in an epidemic and may vary over time {[}26{]}. However,
our use of a bootstrapped subsampling approach mitigates these issues by
allowing multiple delay distributions based on the observed data to be
considered at the cost of increasing uncertainty in our estimates.

Our model is also limited by the data available to us. For example, the
publicly available linelists contain little data on the importation
status of cases. This means that cases counts may be biased upwards by
attributing imported cases to local transmission. This bias is
particularly problematic when case counts are low. Unfortunately, in the
absence of data, this issue can only be explored via scenario analysis.

As more data becomes available, future work should look to refine the
distributions used for generation time, incubation period, and the
report delay. There is also the potential to extend the present model to
account for changes in the delay from onset to notification over the
course of an outbreak though additional data would need to be available
for this to be possible. Finally, there is scope to explore how outbreak
dynamics that differ among particular sub-populations, such as high-risk
COVID-19 patients, can bias overall reproduction number estimates. This
may be achieved by comparing reproduction number estimates from
disparate data sources such as test positive cases, hospital admissions,
and test positive deaths.

Our approach, providing real-time estimates of the reproduction number,
serves as a valuable tool for decision makers looking to track the
course of COVID-19 outbreaks. The nowcasts explicitly account for
delays, using the same methodology across all countries and sub-national
regions. These reproduction number estimates may also be used to
ascertain the likely outbreak trajectory if no policy interventions are
made. They can also provide real-time feedback on whether transmission
is decreasing following a particular intervention, or whether it is
increasing following the relaxing or lifting of current intervention
measures. We hope that our website and the related toolkit will provide
a valuable resource for devising strategies to contain COVID-19
outbreaks worldwide.

\hypertarget{data-availability}{%
\subsection{Data availability}\label{data-availability}}

Latest data: \url{https://dataverse.harvard.edu/dataverse/covid-rt}

Archived data at the time of publication:
\url{https://dataverse.harvard.edu/dataverse/covid-rt}

License: \href{https://opensource.org/licenses/MIT}{MIT}

\hypertarget{software-availability}{%
\subsection{Software availability}\label{software-availability}}

\hypertarget{development}{%
\subsubsection{Development}\label{development}}

\begin{itemize}
\tightlist
\item
  Website (Front-end): \url{https://github.com/epiforecasts/covid}
\item
  Scheduling framework:
  \url{https://github.com/epiforecasts/covid-rt-estimates}
\item
  \emph{EpiNow2} R package (R estimation, data processing, visualisation
  and reporting): \url{https://github.com/epiforecasts/EpiNow2}
\item
  \emph{covidregionaldata} R package (data aggregation and processing):
  \url{https://github.com/epiforecasts/covidregionaldata}
\item
  \emph{RtD3} R package (interative visualisation):
  \url{https://github.com/epiforecasts/RtD3}
\end{itemize}

\hypertarget{archived-at-the-time-of-publication}{%
\subsubsection{Archived at the time of
publication}\label{archived-at-the-time-of-publication}}

\begin{itemize}
\tightlist
\item
  Website: \url{https://doi.org/10.5281/zenodo.3841818}
\item
  Scheduling framework:
  \url{https://github.com/epiforecasts/covid-rt-estimates}
\item
  \emph{EpiNow2} R package {[}10{]}:
  \url{https://doi.org/10.5281/zenodo.3957489}
\item
  \emph{covidregionaldata} R package {[}11{]}:
  \url{https://doi.org/10.5281/zenodo.3957539}
\item
  \emph{RtD3} {[}24{]}: \url{https://doi.org/10.5281/zenodo.4011841}
\end{itemize}

License: \href{https://opensource.org/licenses/MIT}{MIT}

\hypertarget{acknowledgements}{%
\subsection{Acknowledgements}\label{acknowledgements}}

This project was enabled through access to the \textbf{MRC eMedLab
Medical Bioinformatics infrastructure}, supported by the \textbf{Medical
Research Council} (MR/L016311/1). Additional compute infrastructure and
support was provided by the \textbf{Met office}. We thank Venexia Walker
for comments on a version of this draft. The following authors were part
of the Centre for Mathematical Modelling of Infectious Disease 2019-nCoV
working group. Each contributed in processing, cleaning and
interpretation of data, interpreted findings, contributed to the
manuscript, and approved the work for publication: Samuel Clifford, Mark
Jit, Stéphane Hué, Eleanor M Rees, Petra Klepac, Damien C Tully, Rachel
Lowe, Kathleen O'Reilly, Nicholas G. Davies, Quentin J Leclerc, Arminder
K Deol, Gwenan M Knight, C Julian Villabona-Arenas, Fiona Yueqian Sun,
Emily S Nightingale, Alicia Rosello, Adam J Kucharski, Yang Liu, Billy J
Quilty, Matthew Quaife, Jon C Emery, Katherine E. Atkins, Simon R
Procter, W John Edmunds, Megan Auzenbergs, Christopher I Jarvis, David
Simons, Kiesha Prem, Graham Medley, Thibaut Jombart, Charlie Diamond,
Anna M Foss, Rein M G J Houben, Kevin van Zandvoort, Georgia R
Gore-Langton.

\hypertarget{funding}{%
\subsection{Funding}\label{funding}}

The following funding sources are acknowledged as providing funding for
the named authors. Alan Turing Institute (AE). This research was partly
funded by the Bill \& Melinda Gates Foundation (NTD Modelling Consortium
OPP1184344: CABP). DFID/Wellcome Trust (Epidemic Preparedness
Coronavirus research programme 221303/Z/20/Z: CABP). This research was
partly funded by the Global Challenges Research Fund (GCRF) project
`RECAP' managed through RCUK and ESRC (ES/P010873/1: AG). HDR UK
(MR/S003975/1: RME). Nakajima Foundation (AE). UK DHSC/UK Aid/This
research was partly funded by the National Institute for Health Research
(NIHR) using UK aid from the UK Government to support global health
research. The views expressed in this publication are those of the
author(s) and not necessarily those of the NIHR or the UK Department of
Health and Social Care (ITCRZ 03010: HPG). UK MRC (MC\_PC 19065: RME).
Wellcome Trust (206250/Z/17/Z: TWR; 208812/Z/17/Z: SFlasche;
210758/Z/18/Z: JDM, JH, NIB, SA, SFunk, SRM).

\hypertarget{references}{%
\section*{References}\label{references}}
\addcontentsline{toc}{section}{References}

\hypertarget{refs}{}
\leavevmode\hypertarget{ref-Linton:2020gg}{}%
1 Linton NM, Kobayashi T, Yang Y \emph{et al.} Incubation period and
other epidemiological characteristics of 2019 novel coronavirus
infections with right truncation: A statistical analysis of publicly
available case data. \emph{Journal of clinical medicine}
2020;\textbf{9}.

\leavevmode\hypertarget{ref-Cori:2017fg}{}%
2 Cori A, Donnelly CA, Dorigatti I \emph{et al.} Key data for outbreak
evaluation: Building on the ebola experience. \emph{Philosophical
transactions of the Royal Society of London Series B, Biological
sciences} 2017;\textbf{372}.

\leavevmode\hypertarget{ref-Mizumoto:2020ct}{}%
3 Mizumoto K, Kagaya K, Zarebski A \emph{et al.} Estimating the
asymptomatic proportion of coronavirus disease 2019 (covid-19) cases on
board the diamond princess cruise ship, yokohama, japan, 2020.
\emph{Eurosurveillance : bulletin Europeen sur les maladies
transmissibles = European communicable disease bulletin}
2020;\textbf{25}.

\leavevmode\hypertarget{ref-Donker:2011fk}{}%
4 Donker T, Boven M van, Ballegooijen WM van \emph{et al.} Nowcasting
pandemic influenza a/h1n1 2009 hospitalizations in the netherlands.
\emph{European journal of epidemiology} 2011;\textbf{26}:195--201.

\leavevmode\hypertarget{ref-vandeKassteele:2019cn}{}%
5 Kassteele J van de, Eilers PHC, Wallinga J. Nowcasting the number of
new symptomatic cases during infectious disease outbreaks using
constrained p-spline smoothing. \emph{Epidemiology (Cambridge, Mass)}
2019;\textbf{30}:737--45.

\leavevmode\hypertarget{ref-cori2013}{}%
6 Cori A, Ferguson NM, Fraser C \emph{et al.} A new framework and
software to estimate time-varying reproduction numbers during epidemics.
\emph{American Journal of Epidemiology} 2013;\textbf{178}:1505--12.
doi:\href{https://doi.org/10.1093/aje/kwt133}{10.1093/aje/kwt133}

\leavevmode\hypertarget{ref-THOMPSON2019100356}{}%
7 Thompson RN, Stockwin JE, Gaalen RD van \emph{et al.} Improved
inference of time-varying reproduction numbers during infectious disease
outbreaks. \emph{Epidemics} 2019;\textbf{29}:100356.
doi:\href{https://doi.org/https://doi.org/10.1016/j.epidem.2019.100356}{https://doi.org/10.1016/j.epidem.2019.100356}

\leavevmode\hypertarget{ref-Fraser:2007hf}{}%
8 Fraser C. Estimating individual and household reproduction numbers in
an emerging epidemic. \emph{PloS one} 2007;\textbf{2}:e758.

\leavevmode\hypertarget{ref-dataverse}{}%
9 Abbott S, Hickson J, Allen J \emph{et al.} National and subnational
estimates of the time-varying reproduction number of sars-cov-2.
2020.\url{https://dataverse.harvard.edu/dataverse/covid-rt}

\leavevmode\hypertarget{ref-epinow2}{}%
10 Abbott S, Hellewell J, Hickson J \emph{et al.} EpiNow2: Estimate
real-time case counts and time-varying epidemiological parameters.
\emph{-} 2020;\textbf{-}:--.
doi:\href{https://doi.org/10.5281/zenodo.3957489}{10.5281/zenodo.3957489}

\leavevmode\hypertarget{ref-covidregionaldata}{}%
11 Abbott S, Sherratt K, Bevan J \emph{et al.} Covidregionaldata:
Subnational data for the covid-19 outbreak. \emph{-} 2020;\textbf{-}:--.
doi:\href{https://doi.org/10.5281/zenodo.3957539}{10.5281/zenodo.3957539}

\leavevmode\hypertarget{ref-ecdc_data}{}%
12 Disease Prevention EC for, Control. Download today's data on the
geographic distribution of covid-19 cases worldwide. 2020.
\url{www.ecdc.europa.eu/en/publications-data/download-todays-data-geographic-distribution-covid-19-cases-worldwide}

\leavevmode\hypertarget{ref-kraemer2020epidemiological}{}%
13 Xu B, Gutierrez B, Hill S \emph{et al.} Epidemiological data from the
nCoV-2019 outbreak: Early descriptions from publicly available data.
\url{http://virological.org/t/epidemiological-data-from-the-ncov-2019-outbreak-early-descriptions-from-publicly-available-data/337}

\leavevmode\hypertarget{ref-rstan}{}%
14 Stan Development Team. RStan: The r interface to stan.
2020.\url{http://mc-stan.org/}

\leavevmode\hypertarget{ref-Imai:webreport3}{}%
15 Imai N, Cori A, Dorigatti I \emph{et al.} Report 3: Transmissibility
of 2019-nCoV.
\url{https://www.imperial.ac.uk/media/imperial-college/medicine/sph/ide/gida-fellowships/Imperial-2019-nCoV-transmissibility.pdf}

\leavevmode\hypertarget{ref-Abbott:2020hj}{}%
16 Abbott S, Hellewell J, Munday J \emph{et al.} The transmissibility of
novel coronavirus in the early stages of the 2019-20 outbreak in wuhan:
Exploring initial point-source exposure sizes and durations using
scenario analysis. \emph{Wellcome open research} 2020;\textbf{5}:17.

\leavevmode\hypertarget{ref-approxGP}{}%
17 Riutort-Mayol G, Bürkner P-C, Andersen MR \emph{et al.} Practical
hilbert space approximate bayesian gaussian processes for probabilistic
programming. 2020.\url{http://arxiv.org/abs/2004.11408}

\leavevmode\hypertarget{ref-generationinterval}{}%
18 Ganyani T, Kremer C, Chen D \emph{et al.} Estimating the generation
interval for coronavirus disease (covid-19) based on symptom onset data,
march 2020. \emph{Eurosurveillance} 2020;\textbf{25}.

\leavevmode\hypertarget{ref-incubationperiod}{}%
19 Lauer SA, Grantz KH, Bi Q \emph{et al.} The incubation period of
coronavirus disease 2019 (covid-19) from publicly reported confirmed
cases: Estimation and application. \emph{Annals of Internal Medicine}
2020;\textbf{172}:577--82.

\leavevmode\hypertarget{ref-Park2019}{}%
20 Park SW, Champredon D, Weitz JS \emph{et al.} A practical
generation-interval-based approach to inferring the strength of
epidemics from their speed. \emph{Epidemics} 2019;\textbf{27}:12--8.
doi:\href{https://doi.org/https://doi.org/10.1016/j.epidem.2018.12.002}{https://doi.org/10.1016/j.epidem.2018.12.002}

\leavevmode\hypertarget{ref-Russell:BFVkJ6lQ}{}%
21 Russell TW. Using a delay-adjusted case fatality ratio to estimate
under-reporting.
\url{https://cmmid.github.io/topics/covid19/severity/global_cfr_estimates.html}

\leavevmode\hypertarget{ref-rmarkdown}{}%
22 Xie Y, Allaire JJ, Grolemund G. \emph{R markdown: The definitive
guide}. Boca Raton, Florida:: Chapman; Hall/CRC 2018.
\url{https://bookdown.org/yihui/rmarkdown}

\leavevmode\hypertarget{ref-distill}{}%
23 Allaire J, Iannone R, Xie Y. \emph{Distill: R markdown format for
scientific and technical writing}. 2020.
\url{https://github.com/rstudio/distill}

\leavevmode\hypertarget{ref-rtd3}{}%
24 Gibbs H, Abbott S, Funk S. RtD3: Rt visualization in d3.
\emph{Zenodo} 2020;\textbf{-}:--.
doi:\href{https://doi.org/10.5281/zenodo.4011841}{10.5281/zenodo.4011841}

\leavevmode\hypertarget{ref-rt-comparison}{}%
25 Sherratt K, Abbott S, Meakin SR \emph{et al.} Evaluating the use of
the reproduction number as an epidemiological tool, using
spatio-temporal trends of the covid-19 outbreak in england.
\emph{medRxiv} Published Online First: 2020.
doi:\href{https://doi.org/10.1101/2020.10.18.20214585}{10.1101/2020.10.18.20214585}

\leavevmode\hypertarget{ref-Britton:2019gf}{}%
26 Britton T, Scalia Tomba G. Estimation in emerging epidemics: Biases
and remedies. \emph{Journal of the Royal Society, Interface}
2019;\textbf{16}.

\end{document}
